\subsection*{Information}


\begin{DoxyItemize}
\item {\bfseries Brief}\+: Canopen object able to send command through a C\+AN interface using the U\+N\+IX socket.
\item {\bfseries Languages}\+: C++
\item {\bfseries Libraries}\+:
\item {\bfseries Note}\+: /
\item {\bfseries Compatibility}\+:
\end{DoxyItemize}

\tabulinesep=1mm
\begin{longtabu} spread 0pt [c]{*{3}{|X[-1]}|}
\hline
\rowcolor{\tableheadbgcolor}\PBS\centering \textbf{ Ubuntu }&\PBS\centering \textbf{ Window10 }&\PBS\centering \textbf{ Mac\+OS  }\\\cline{1-3}
\endfirsthead
\hline
\endfoot
\hline
\rowcolor{\tableheadbgcolor}\PBS\centering \textbf{ Ubuntu }&\PBS\centering \textbf{ Window10 }&\PBS\centering \textbf{ Mac\+OS  }\\\cline{1-3}
\endhead
\PBS\centering \+:heavy\+\_\+check\+\_\+mark\+:&\PBS\centering \+:grey\+\_\+question\+:&\PBS\centering \+:grey\+\_\+question\+: \\\cline{1-3}
\end{longtabu}


\subsection*{Building}

\subsubsection*{Ubuntu}

\paragraph*{Steps}


\begin{DoxyItemize}
\item Clone the repository and go inside. 
\begin{DoxyCode}
git clone https://gitlab-dev.isir.upmc.fr/devillard/canopen.git && cd canpen
\end{DoxyCode}

\item Create a build directory and go inside.
\item Configure the project.
\item Build the project. 
\begin{DoxyCode}
mkdir build && cd build && cmake .. && cmake --build .
\end{DoxyCode}

\end{DoxyItemize}

\subsection*{Canopen program}

The executable file canopen enable you to send S\+DO message to a C\+AN bus. \subsubsection*{usage\+:}

./canopen ifname 0xindex 0xsub \mbox{[} size base data \mbox{]}

Arg\+: ifname \+: C\+AN interface name 0xindex \+: Object register index 0xsub \+: Object register subindex size \+: Data size (number of bytes) base \+: Numerical base of the value passed. data \+: Value to write.

Ex\+: To read register 0x1000\+:2 of node 4 on \char`\"{}can0\char`\"{}\+: ./canopen can0 4 1000 2

To write in register 0x2000\+:F of node 3 the value 0x1234 on \char`\"{}can0\char`\"{}\+: ./canopen can0 3 2000 F 2 x 1234

\subsection*{Library Methods}

\#\#\#\# P\+DO 
\begin{DoxyCode}
\textcolor{keywordtype}{void} set\_PDO<N,M>(uint8\_t ID); \textcolor{comment}{// Set the R/W PDO numero N. M: SDO\_W / SDO\_R}
\end{DoxyCode}
 
\begin{DoxyCode}
\textcolor{keywordtype}{void} send\_PDO<N>(uint8\_t ID, T data); \textcolor{comment}{// Send the data of type T (anytype) with the PDO N.}
\end{DoxyCode}
 
\begin{DoxyCode}
\textcolor{keywordtype}{void} recv(T& data); \textcolor{comment}{// Receive the data of type T (anytype).}
\end{DoxyCode}


\#\#\#\# S\+DO 
\begin{DoxyCode}
\textcolor{keywordtype}{void} send\_SDO(uint8\_t ID, uint16\_t index, uint8\_t sub, T data); \textcolor{comment}{// Send the data of type T (anytype) with a
       SDO message at index:sub.}
\end{DoxyCode}
 \paragraph*{N\+MT}


\begin{DoxyCode}
\textcolor{keywordtype}{void} send\_NMT(uint16\_t msg); \textcolor{comment}{// Send a NMT message. NMT\_START / NMT\_STOP / NMT\_PREOP / NMT\_RESET\_NODE /
       NMT\_RESET\_COMM}
\end{DoxyCode}
 